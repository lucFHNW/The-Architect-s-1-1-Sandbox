\section{Interviewauswertung mit dem Kunden – Feldtest}

Im Anschluss an den Feldtest wurde ein qualitatives Interview mit einem Vertreter des SCDH durchgeführt. Ziel war es, die Perspektive der Auftraggeberseite hinsichtlich Integration, Nutzen und möglicher Weiterentwicklungen des Systems zu erfassen.

\subsection{Integration in bestehende Workshop-Formate}

\textbf{Frage:} Wie gut lässt sich das aktuelle System in Ihre bestehenden Workshop-Formate integrieren?

\textbf{Antwort:}
\begin{itemize}
    \item Das System bewegt sich technisch in die richtige Richtung.
    \item Die Integration mit bestehenden PDF-Workflows funktioniert bereits gut.
    \item Details zum Workshop-Ablauf müssen noch abgestimmt werden.
    \item Insgesamt zeigt sich ein vielversprechendes Potenzial mit klar reaktiver Nutzung im Vergleich zu bestehenden Methoden.
\end{itemize}

\subsection{Grösster Nutzen für den Arbeitsalltag}

\textbf{Frage:} Wo sehen Sie den grössten Nutzen dieses Systems für Ihre tägliche Arbeit?

\textbf{Antwort:}
\begin{itemize}
    \item Möglichkeit, Änderungen am Plan reaktiv vorzunehmen – grosser Gewinn.
    \item Arbeit mit einer leeren Zeichenfläche (vergleichbar mit weissem Papier) ermöglicht kreative Freiheit.
\end{itemize}

\subsection{Potenzial für andere Zielgruppen}

\textbf{Frage:} Welche Anwendungsfälle oder Zielgruppen könnten von einer Weiterentwicklung profitieren?

\textbf{Antwort:}
\begin{itemize}
    \item Architekt:innen, da viele Kommunikationsschleifen (E-Mails, Telefonate) wegfallen.
    \item Stadtverwaltung (z.B. Planung öffentlicher Räume, Parkplätze).
    \item Brainstorming-Tool oder technische Planungen (z.B. Elektriker:innen, Techniker:innen).
\end{itemize}

\subsection{Voraussetzungen für produktiven Einsatz}

\textbf{Frage:} Welche Voraussetzungen müsste das System erfüllen, um produktiv eingesetzt werden zu können?

\textbf{Antwort:}
\begin{itemize}
    \item Unterstützung mehrerer Displays und Spiegelfunktion.
    \item Zuverlässige und exakte Kalibrierung (1:1).
    \item Durchdachtes Setup (Installationsort, Abläufe).
    \item Aktuelle Funktionen sind bereits brauchbar für Workshops.
    \item Einsatz von zwei IR-Kameras könnte Vorteile bringen.
    \item Import von Plänen ist essenziell und funktioniert gut.
\end{itemize}

\subsection{Gewünschte Funktionen und Erweiterungen}

\textbf{Frage:} Welche Funktionen oder Erweiterungen wünschen Sie sich für eine zukünftige Version?

\textbf{Antwort:}
\begin{itemize}
    \item QR-Codes zur Visualisierung von Objekten (z.B. Bett, Stuhl).
    \item Objektkatalog zur schnellen Auswahl und Platzierung.
    \item Möglichkeit zur Remote-Zusammenarbeit.
    \item Reset-Buttons für 0\% und 100\%-Zoom.
    \item Undo-/Redo-Funktion.
    \item Kommentarfunktion via Post-its oder Textfelder.
\end{itemize}

\subsection{Herausforderungen bei der Einführung}

\textbf{Frage:} Welche Herausforderungen sehen Sie bei der Einführung eines solchen Systems in der Praxis?

\textbf{Antwort:}
\begin{itemize}
    \item Hardwarebeschaffung (Stift, Projektor etc.).
    \item Planung des Setups: Wo und wie wird aufgebaut?
    \item Integration in bestehende Softwareumgebungen könnte komplex werden.
\end{itemize}

\subsection{Weitere Rückmeldungen}

\textbf{Frage:} Gibt es sonstige Rückmeldungen oder Verbesserungsvorschläge, die Sie uns mitgeben möchten?

\textbf{Antwort:}
\begin{itemize}
    \item Keine weiteren Vorschläge.
    \item Positives Feedback: Die Mitwirkung des Projektteams wurde sehr geschätzt.
\end{itemize}