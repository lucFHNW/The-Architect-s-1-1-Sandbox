\section{Einleitung}
\label{einleitung}

Das Projekt \textit{The Architect's 1:1 Sandbox} wird im Auftrag des Swiss Center for Design and Health (SCDH) in Nidau durchgeführt. Das SCDH betreibt an seinem Standort in Nidau bei Biel eine Einrichtung, in der mittels Projektionen und Mockups Gebäudepläne im Massstab 1:1 dargestellt werden können. Diese Technologie ermöglicht die Erprobung von Abläufen und Prozessen sowie die Überprüfung von Designs in einer möglichst realitätsnahen Umgebung. Um dieses Angebot zu erweitern, soll im Rahmen des Projekts eine Softwarelösung entwickelt werden, die den initialen Gebäudedesign-Prozess vereinfacht und verbessert.

Ein zentrales Problem besteht darin, dass viele Personen keine Erfahrung im Gestalten von Gebäuden haben und dadurch während der ersten Bedarfsanalyse Schwierigkeiten bei der Formulierung und Visualisierung ihrer Anforderungen und Wünsche erleben. Das SCDH benötigt daher eine benutzerfreundliche Software, die es ermöglicht, gemeinsam mit den Kunden grobe Gebäudepläne zu erstellen und diesen ein besseres Verständnis sowie eine klare Übersicht über ihre Planung zu vermitteln.

Es existieren bereits einige CAD-Softwarelösungen zum Erstellen von Gebäudeplänen. Diese sind jedoch primär auf technische Detailplanung ausgerichtet und setzen gewisse Vorkenntnisse voraus. Zudem wurden sie selten für kollaborative Nutzung konzipiert, sondern meist für individuelle Arbeitsprozesse.

Das Ziel dieses Projekts ist die Entwicklung einer Softwarelösung, die auch Personen ohne Fachkenntnisse befähigt, funktionale Grundrissanordnungen zu entwerfen und dabei ein verständliches sowie anschauliches Bild der geplanten Räume zu erhalten. Die Zielgruppe ist somit relativ breit gefächert und umfasst alle, die ein Interesse an einer verbesserten Kommunikation und Visualisierung haben. Im Vordergrund steht eine benutzerfreundliche Gestaltung der Anwendung, die eine intuitive Bedienung ermöglicht und den Nutzenden in Kombination mit der 1:1-Projektionsfläche ein realistisches Raumgefühl vermittelt.

Die zu entwickelnde Software soll eine klare und nachvollziehbare Benutzeroberfläche bieten, die keine spezifischen architektonischen Vorkenntnisse voraussetzt. Sie soll es ermöglichen, individuelle Raumkonzepte mit Fokus auf Raumabfolgen und funktionale Zusammenhänge zu skizzieren. Dabei wird auf einfache Navigation, logische Strukturierung der Funktionen und unmittelbare Rückmeldung bei Eingaben besonderer Wert gelegt.

Ein wesentliches Merkmal der Lösung ist die Möglichkeit, Änderungen in Echtzeit vorzunehmen, die unmittelbar auf der 1:1-Projektionsfläche visualisiert werden. Dies soll auch eine kollaborative Arbeitsweise fördern, in der verschiedene Beteiligte wie Architekt:innen und Endnutzer:innen gemeinsam Planungen diskutieren und anpassen können.

Aus dieser Zielsetzung ergibt sich folgende Aufgabenstellung: Es soll eine flexible, intuitive und kollaborative Zeichenumgebung entstehen, die es ermöglicht, Bauelemente und Raumstrukturen in Echtzeit zu erfassen, zu bearbeiten und direkt zu projizieren. Eine mögliche Interaktionsvariante über ein kleineres Tischsystem mit synchronisierter Verbindung zur grossen Projektionsfläche wird ebenfalls in Betracht gezogen.

\clearpage

Im Rahmen des Projekts werden dazu drei zentrale \textbf{Forschungsfragen} untersucht, die die inhaltliche und gestalterische Entwicklung sowie die abschliessende Evaluation leiten:

\begin{enumerate}
    \item Wie können der Arbeitsablauf und das technische System gestaltet werden, damit Benutzer:innen kollaborativ an der einfachen und effizienten Planung von Grundrissen arbeiten können?
    \item Wie kann die Benutzeroberfläche so gestaltet werden, dass sie flexibel und funktionsreich ist, gleichzeitig aber auch für unerfahrene Nutzer:innen verständlich und intuitiv bleibt?
    \item Wie wird die vorgeschlagene Lösung von potenziellen Benutzer:innen hinsichtlich ihrer Benutzerfreundlichkeit und der Förderung der Zusammenarbeit wahrgenommen?
\end{enumerate}

Insgesamt zielt das Projekt darauf ab, eine benutzerzentrierte Lösung zu entwickeln, die die Komplexität der Gebäudeplanung reduziert, gleichzeitig aber ausreichend Flexibilität bietet, um individuelle Vorstellungen und Anforderungen umzusetzen. Dadurch soll die Software sowohl für private Bauherr:innen als auch für andere interessierte Gruppen ein nützliches Werkzeug zur interaktiven Planung und Visualisierung von Gebäuden darstellen.
