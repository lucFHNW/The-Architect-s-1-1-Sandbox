\section*{\centering{Abstract}}
Diese Bachelorarbeit beschreibt die Entwicklung eines digitalen Werkzeugs zur kollaborativen Grundrissgestaltung auf einer 1:1-Projektionsfläche. Ziel war es, eine benutzerfreundliche Anwendung zu realisieren, die insbesondere Laien in partizipativen Architektur-Workshops des Swiss Center for Design and Health (SCDH) unterstützt. Kern der Lösung ist ein interaktives Zeichenwerkzeug mit Infrarotstift-Tracking, dessen Position über eine Kamera erfasst und in Echtzeit auf die Projektionsfläche übertragen wird. Die Anwendung ermöglicht es mehreren Personen, simultan Raumkonzepte zu skizzieren, anzupassen und unmittelbar zu visualisieren.

Die plattformunabhängige Systemarchitektur wurde in praxisnahen Tests und einem Feldtest am SCDH evaluiert. Die Ergebnisse, darunter ein durchschnittlicher System Usability Scale (SUS)-Wert von 78.33 Punkten, bestätigen die intuitive Bedienbarkeit und den hohen Mehrwert für kollaborative Planungsprozesse. Damit leistet die Arbeit einen Beitrag zur Entwicklung niederschwelliger, technisch robuster Interfaces für die gemeinsame Raumplanung.

