\section{Analyse der Ausgangslage}
In diesem Kapitel werden die Probleme und Anforderungen behandelt, die sich aus der Aufgabenstellung sowie unseren Beobachtungen während des Workshops ergeben haben.

\subsection{Anforderungen und Probleme aus der Aufgabenstellung}
Basierend auf der in Kapitel 2.1 vorgestellten Aufgabenstellung haben wir folgende zentrale Anforderungen identifiziert:

\begin{itemize}
  \item Echtzeit-Projektion auf die 1:1-Projektionsfläche des SCDH.
  \item Ermöglichung einer visuellen, interaktiven Planung (Hands-on-Ansatz).
  \item Fokus auf Zusammenarbeit und Klarheit in der Kommunikation.
  \item Die Lösung soll ein Hilfsmittel sein, mit dem auch Laien ihre Platzbedürfnisse erkunden und verständlich kommunizieren können.
\end{itemize}

\subsection{Anforderungen und Probleme aus dem Workshop}
Im Rahmen des in Kapitel 2.2 beschriebenen Workshops konnten wir die Anforderungen aus der Aufgabenstellung bestätigen. Zudem liessen sich auf Grundlage unserer Beobachtungen weitere Anforderungen ableiten:

\begin{itemize}
  \item Die Lösung sollte auch für Personen mit geringen technischen Kenntnissen oder sprachlichen Barrieren einsetzbar sein.
  \item Die Interaktion sollte möglichst intuitiv erfolgen, um Diskussionen im Team nicht zu unterbrechen oder zu hemmen.
\end{itemize}

\subsection{Forschungsfragen}
Aus der Kombination von Aufgabenstellung und Workshop-Erkenntnissen haben wir drei zentrale Forschungsfragen abgeleitet. Diese bilden die Grundlage für unsere Designentscheide und die konzeptionelle Ausrichtung des Projekts:

\begin{itemize}
  \item \textit{Wie können der Arbeitsablauf und das technische System gestaltet werden, damit Benutzerinnen und Benutzer kollaborativ an der einfachen und effizienten Planung von Grundrissen arbeiten können?}
  \item \textit{Wie kann die Benutzeroberfläche so gestaltet werden, dass sie flexibel und funktionsreich ist, gleichzeitig aber auch für unerfahrene Nutzerinnen und Nutzer verständlich und intuitiv bleibt?}
  \item \textit{Wie wird die vorgeschlagene Lösung von potenziellen Benutzerinnen und Benutzern hinsichtlich ihrer Benutzerfreundlichkeit und der Förderung der Zusammenarbeit wahrgenommen?}
\end{itemize}
