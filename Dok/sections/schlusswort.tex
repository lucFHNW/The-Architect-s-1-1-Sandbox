\section{Schlusswort}

Im Rahmen dieses Projekts wurde eine interaktive Zeichenanwendung entwickelt, die es Laien ermöglicht, auf einer 1:1-Projektionsfläche funktionale Raumkonzepte kollaborativ zu entwerfen und direkt zu erleben.  
Die Lösung wurde bewusst benutzerzentriert konzipiert und in enger Abstimmung mit realen Anwendungsbedürfnissen, insbesondere im Kontext von partizipativen Architektur-Workshops am SCDH, umgesetzt.  

Durch den Einsatz eines Infrarotstifts und einer kamerabasierten Erkennung konnte eine präzise, flexible und kostengünstige Eingabemethode realisiert werden.  
Die entwickelte Software überzeugte sowohl in den Usability-Tests, mit einem durchschnittlichen SUS-Wert von 78.33 Punkten deutlich über dem branchenüblichen Schwellenwert, als auch im Feldtest mit dem Kunden, der die Praxistauglichkeit unter realistischen Bedingungen bestätigte.  

Auch wenn nicht alle Funktionen vollständig ausgereift oder umgesetzt wurden, wie etwa automatische Kalibrierung, Undo/Redo oder eine softwareseitige Mehrbenutzerverwaltung, zeigt das Ergebnis klar das Potenzial für den Praxiseinsatz.  
Die Forschungsfragen wurden weitgehend beantwortet und die Lösung bildet eine solide Grundlage für weiterführende Arbeiten, etwa zur Evaluation unter realen Bedingungen oder zur Integration zusätzlicher Funktionen wie Objektbibliotheken, erweiterten Exportoptionen oder KI-gestützter Planungshilfen.  

Rückblickend war das Projekt nicht nur eine technische, sondern auch eine methodische und kommunikative Herausforderung, bei der technische Machbarkeit, Nutzerbedürfnisse und gestalterische Entscheidungen stets im Gleichgewicht gehalten werden mussten.  
Die enge Zusammenarbeit mit dem SCDH und die praxisnahe Einbettung ermöglichten es, eine Lösung zu schaffen, die bereits heute einen echten Mehrwert bietet.  
Mit den gewonnenen Erkenntnissen und den identifizierten Verbesserungsmöglichkeiten ist der Grundstein gelegt, um das System in zukünftigen Versionen zu einem noch leistungsfähigeren, flexibleren und vielseitigeren Werkzeug für die kollaborative Raumplanung auszubauen.  

