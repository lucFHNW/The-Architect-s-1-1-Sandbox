\section{Diskussion}

In diesem Kapitel werden die Ergebnisse und Beobachtungen im Rahmen des Projekts kritisch reflektiert. Dabei stehen sowohl die Einordnung der Lösung in den Anwendungskontext als auch ihre Stärken, Schwächen und offene Fragen im Vordergrund.

\subsection{Erfüllung der Projektziele}

Die entwickelte Anwendung erfüllt die grundlegenden Anforderungen, die in der Aufgabenstellung und in den Workshops definiert wurden.  
Nutzer:innen können mit dem IR-Stift intuitiv auf einer Projektionsfläche zeichnen, Inhalte in Echtzeit anpassen und gemeinsam an Raumkonzepten arbeiten.  
Besonders positiv hervorzuheben ist die niedrige Einstiegshürde für Laien sowie die plattformübergreifende Architektur.

Die im Rahmen der Usability-Tests erreichten Ergebnisse bestätigen den Erfolg der Umsetzung: Mit einem durchschnittlichen SUS-Wert von 78.33 Punkten liegt die Anwendung deutlich über dem branchenüblichen Schwellenwert von 68 Punkten und weist somit eine hohe Gebrauchstauglichkeit auf.  
Auch der durchgeführte Feldtest mit dem Kunden hat die Praxistauglichkeit der Lösung unter realistischen Bedingungen bestätigt und wertvolle Hinweise zur Weiterentwicklung geliefert.  
Diese Kombination aus hoher Benutzerfreundlichkeit und positiver Rückmeldung aus der realen Anwendung belegt, dass die gesteckten Projektziele nicht nur erfüllt, sondern in wesentlichen Aspekten übertroffen wurden.


\subsection{Einordnung in den Anwendungskontext}

Der Einsatz im Workshop-Kontext am SCDH hat sich als besonders sinnvoll erwiesen.  
Die Lösung adressiert reale Herausforderungen wie Kommunikationsbarrieren, unklare Türrichtungen oder mangelnde Flexibilität bei kurzfristigen Änderungen.  
Durch die direkte Visualisierung auf der 1:1-Projektionsfläche können Anpassungen ohne Zeitverlust umgesetzt und allen Teilnehmenden sofort präsentiert werden.

Ein wesentlicher Vorteil für das SCDH liegt in der hohen Flexibilität des Systems.  
Es lässt sich sowohl mit bestehenden Grundrissplänen als auch auf einer leeren Projektionsfläche einsetzen und unterstützt so unterschiedliche Workshop-Formate von der frühen Konzeptphase bis zur detaillierten Layoutdiskussion.  
Zudem kann das System schnell auf neue Szenarien angepasst werden, ohne dass zusätzliche Schulungen oder technisches Vorwissen erforderlich sind.  

Der Feldtest hat gezeigt, dass diese Flexibilität in der Praxis von grossem Vorteil ist.  
So können Änderungen während des Workshops sofort als Bild exportiert und den Architekt:innen zur weiteren Bearbeitung bereitgestellt werden, ohne den Umweg über einen schriftlichen Bericht gehen zu müssen.  
Diese Eigenschaften ermöglichen es dem SCDH, die Anwendung nicht nur in partizipativen Workshops mit Laien, sondern auch in Projekten mit Architekt:innen, Planungsämtern und weiteren Fachdisziplinen einzusetzen.

\clearpage

\subsection{Reflexion der Forschungsfragen}

Die im Projekt formulierten Forschungsfragen lassen sich rückblickend wie folgt einordnen:

\begin{itemize}
    \item \textbf{Forschungsfrage 1} (Systemgestaltung für kollaborative Planung): konnte klar beantwortet werden. Durch den Einsatz des IR-Stifts, die Echtzeitprojektion und die Möglichkeit der gleichzeitigen Nutzung durch mehrere Personen wurde eine effektive kollaborative Arbeitsweise ermöglicht. Sowohl die Usability-Tests als auch der Feldtest haben bestätigt, dass der Arbeitsfluss deutlich verbessert und Missverständnisse reduziert werden konnten.

    
    \item \textbf{Forschungsfrage 2} (intuitive Benutzeroberfläche für Laien): die Tests zeigen, dass die Stift-Metapher und das visuelle Feedback ohne zusätzliche Schulung verstanden werden. Die hohe durchschnittliche Bewertung von 78.33 Punkten im SUS unterstreicht, dass die Bedienung auch für Erstnutzende zugänglich und leicht verständlich ist. Beobachtungen bei den Tests zeigten, dass selbst Personen ohne technische Erfahrung nach kurzer Einweisung eigenständig arbeiten konnten.
    
    \item \textbf{Forschungsfrage 3} (Usability in realen Szenarien): der Feldtest mit dem Kunden hat gezeigt, dass die Lösung im praktischen Einsatz nicht nur angenommen, sondern aktiv als hilfreiches Werkzeug in Workshops wahrgenommen wird. Besonders die Fähigkeit, flexibel auf neue Anforderungen zu reagieren und Ergebnisse sofort zu visualisieren, wurde positiv hervorgehoben.
\end{itemize}


\subsection{Grenzen der Lösung}

Die Ergebnisse der Usability-Tests haben deutlich gemacht, dass für einen produktiven Einsatz zusätzliche Funktionen erforderlich sind und bestehende Funktionen in ihrer Umsetzung weiter optimiert werden sollten.  
Zwar unterstützt das System die Arbeit mit mehreren Stiften gleichzeitig, jedoch können diese aktuell nicht voneinander unterschieden werden. Dies bedeutet, dass alle Stifte dieselben Einstellungen verwenden.  

Aus technischer Sicht ist das System zudem durch die Verwendung einer einzelnen Kamera limitiert.  
Für eine zuverlässigere Erkennung wäre der Einsatz von mindestens einer weiteren Kamera sinnvoll, um zu verhindern, dass die Handhaltung die Stiftspitze verdeckt und dadurch die Erfassung unterbrochen wird.

\clearpage

\subsection{Offene Fragen und Ausblick}

Einige offene Fragen konnten im Rahmen des Projekts nur teilweise beantwortet werden und bilden Ansatzpunkte für zukünftige Arbeiten.  
Dazu gehört insbesondere die Frage, wie sich die Usability bei mehreren aktiven Nutzer:innen gleichzeitig verhält.  
Hier könnte eine optimierte Eingabeverwaltung helfen, Überschneidungen und unbeabsichtigte Interaktionen zu minimieren.

Aus dem Feldtest gingen zudem konkrete Funktionswünsche hervor:  
Eine Undo/Redo-Funktion, ein integrierter Objektkatalog, erweiterte Exportmöglichkeiten (z. B. PDF-Export oder direkter Versand an Projektbeteiligte) sowie eine präzisere Kalibrierung wurden mehrfach genannt.  
Auch die Möglichkeit zur Unterstützung von Multi-Display-Setups und Remote-Zusammenarbeit könnte den Einsatzbereich erheblich erweitern, etwa für hybride Workshops oder parallele Planungssitzungen.  

Darüber hinaus wäre die Integration zusätzlicher Visualisierungsmöglichkeiten wie farbliche Layer oder Objektschatten ein potenzieller Mehrwert,  
um komplexere Entwürfe übersichtlicher darzustellen.  
Langfristig könnte auch eine Cloud-Anbindung zur Speicherung und Nachverfolgung von Workshop-Ergebnissen sowie eine Tablet-basierte Steuerung  
zur Ergänzung der Projektion implementiert werden.  

Diese Weiterentwicklungen würden nicht nur die Funktionalität und Flexibilität des Systems erhöhen,  
sondern auch seine Attraktivität für weitere Zielgruppen wie Architekturbüros, städtische Planungsämter oder technische Gewerke deutlich steigern.
