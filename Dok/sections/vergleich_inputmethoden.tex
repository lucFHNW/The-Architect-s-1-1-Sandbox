
\section{Vergleich der unterschiedlichen Inputmethoden}

In diesem Kapitel werden die Verschieden Input methoden die betrachted wurden anhand der Researchquestions evaluiert um die Wahl der IR variante zu begründen.

\subsection{Lusee}

Bei Lusee handelt es sich um ein Spinoff der FHNW. Lusee ist ein Projektionssystem mit Kamera welches in der Lage ist Finger zu tracken und als Input zu verwenden. Wir haben für den Vergleich die Version von Lusee, aus dem Institut für interaktive Technologien(IIT) der FHNW genommen. Die Version von Lusee ist damit nicht auf dem aktuellsten Stand und es ist möglich das neuere Versionen von uns gefunden Probleme bereits behoben haben. 

\textbf{Vorteile}

Lusee würde uns eine sehr einzig artige Input Möglichkeit bieten mittels dem Fingertracking.

\textbf{Nachteile}

Bei einer ersten erprobung mit der Version von Lusee im IIT hat sich gezeigt das dass System eine merkliche Latenz hat. 
Eine Implementierung für dieses System erhöht die Komplexität des Projektes deutlich ohne einen signifikate verbesserung zu bringen.
Das Malen mit der Hand ist weniger intuitiv als mit einem herkömlichen Stift.

\subsection{IR-Pen}

Beim Infrarotpen(IR-Pen) wird ein Stift welcher über einen Infrarot Emitter verfügt mittels einer Infrarotkamera verfolgt wird. Dabei wird durch Methoden der Bildverarbeitung aus dem Punkt der durch den Emitter von der Kamera aufgenommen wird auf einen Punkt reduziert und kann so als Input verwendet werden.

\textbf{Vorteile}

Ein IR-Pen bietet dem Nutzer eine ihm vertraute Art Zeichnungen und Skizen zu erstellen mittels einem Stift. 
Die Verwendung von IR-Cameras macht das ganze Setup sehr flexible und einfach portierbar.
Eine IR-Pen Version kann sehr kosten günstig implementiert werden.

\textbf{Nachteile}

Eine IR-Pen version hat einen bedeutend höheren Aufwand als die sehr ähnliche Touchscreen version.

\subsection{Touchscreen}

Beim Touchscreen handelt es sich um die allgemein bekannte Inputmethode des Touchscreens in einem Tischformat.

\textbf{Vorteile}

Sehr einfach Implementierung.
Sehr geringe Hardware dependency.
Klassischer Stift Input support.

\textbf{Nachteile}

Hohe Kosten für genügend grossen Touchscreen.
Geringe felxibilität.
