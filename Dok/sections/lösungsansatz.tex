\section{Lösungsansatz}
In diesem Kapitel erläutern wir die Vision unserer Lösung sowie ein Verbesserungspotenzial im Workshop-Ablauf, das durch den Einsatz dieser Lösung adressiert werden kann.

\subsection{Vision}
Unsere Vision ist die Entwicklung eines Systems, das auf einem Tisch platziert oder selbst als Tisch verwendet werden kann. Dieses System befindet sich im Zentrum der Diskussionsrunde, wodurch es allen Beteiligten ermöglicht wird, aktiv daran teilzunehmen. Mithilfe der Visualisierungslösung können Ideen und Konzepte schnell skizziert und so visuell ausgetauscht werden.\\

Zu diesem Zweck haben wir verschiedene Eingabemethoden ausprobiert und evaluiert (siehe Kapitel~5). Unsere Wahl fiel auf eine Variante mit einem Infrarot-Stift. Bei dieser Lösung wird ein Tisch aufgestellt, über dem ein Beamer sowie eine Infrarot-Kamera angebracht sind. Die Benutzerinnen und Benutzer können direkt auf der Tischfläche mit dem Infrarot-Stift zeichnen (detaillierte Beschreibung in Kapitel~7).\\

Diese Eingabemethode ist aus unserer Sicht besonders geeignet, da sie eine sehr intuitive Bedienung ermöglicht, keine umfassende Einführung erfordert und auch von technisch weniger versierten Personen problemlos genutzt werden kann.

\subsection{Verbesserungen in der User Story}
Beim Besuch des Workshops, beschrieben in Kapitel~2.2, konnten wir zwei exemplarische Situationen beobachten, in denen unsere Lösung klare Vorteile bringen würde:

\begin{itemize}
    \item Ein Teilnehmer wollte ein Problem mit der Öffnungsrichtung des Beistosses einer zweiflügeligen Türe erläutern. Das Anliegen wurde vom SCDH-Mitarbeitenden jedoch zunächst nicht verstanden. Eine für alle zugängliche Visualisierung hätte dem Teilnehmer ermöglicht, das Problem direkt im Plan einzuzeichnen – was zu mehr Klarheit und besserer Zusammenarbeit geführt hätte.
    
    \item Während des Debriefings musste die Architektin eine Änderung im Plan einzeichnen. Dazu musste sie aufstehen, nach vorne gehen und sich in einer unbequemen Haltung über den Plan beugen. Dieser Ablauf war ineffizient und unterbrach nach unserem Eindruck den Diskussionsfluss deutlich. Mit unserer Lösung hätte die Architektin die Änderung direkt und bequem von ihrem Platz aus vornehmen können. Auch spätere Anpassungen könnten so einfacher und effizienter durchgeführt werden.
\end{itemize}

\includegraphics[]{userstory_workshop_verbessert.png}
\pagebreak

\subsection{Verknüpfung der Forschungsfragen mit den identifizierten Verbesserungen}
Die im vorherigen Abschnitt geschilderten Situationen zeigen exemplarisch, wie unsere Lösung zur Beantwortung der Forschungsfragen beiträgt:

\vspace{0.5em}
\textit{Wie können der Arbeitsablauf und das technische System gestaltet werden, damit Benutzerinnen und Benutzer kollaborativ an der einfachen und effizienten Planung von Grundrissen arbeiten können?}\\
In beiden Beispielen würde unsere Lösung den Arbeitsablauf vereinfachen und die Kommunikation verbessern, indem sie den direkten und visuellen Austausch komplexer räumlicher Ideen ermöglicht.

\vspace{0.5em}
\textit{Wie wird die vorgeschlagene Lösung von potenziellen Benutzerinnen und Benutzern hinsichtlich ihrer Benutzerfreundlichkeit und der Förderung der Zusammenarbeit wahrgenommen?}\\
In beiden Fällen wurde deutlich, dass die fehlende Möglichkeit zur gemeinsamen visuellen Arbeit ein zentrales Problem darstellt. Unsere Lösung adressiert genau diesen Mangel und bestätigt somit den Bedarf nach einem entsprechenden kollaborativen Werkzeug.
