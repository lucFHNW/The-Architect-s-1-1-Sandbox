\section{Ausgangslage}
In diesem Kapitel wird die Ausgangslage des Projekts beschrieben. Zuerst wird die ursprüngliche Aufgabenstellung erläutert, anschliessend der Ablauf eines Workshops, den wir beim SCDH besuchen konnten.

\subsection{Aufgabenstellung}
Wir sind mit folgender Ausgangslage in das Projekt gestartet:\\
\\
\textit{
"The SCDH wants to develop a solution that allows the users of the projection area to quickly draw floor plans as a brainstorming tool. To accomplish that, it must be possible to create walls and define doors as well as windows. Additionally, a wide range and extensible set of objects must be provided that can be placed and moved interactively on the floor plan. It is important that all modifications of the floor plan are projected on a scale of 1:1 on the projection area in real-time. Finally, the proposed solution should encourage collaboration and be easy to use, also for non-technical users. One possible way to realize the idea is to use another small-scale projection-based interaction system for a tabletop where users can collaborate, interact and place objects to explore design variations. Then, any change on the tabletop would be applied to the 1:1 projection area. This is just one possibility, and we encourage other ideas to connect user input to the 1:1 projection area. The concepts developed will be demonstrated and evaluated as part of a prototype implementation."
}
\\
\\
Diese Aufgabenstellung wurde im Rahmen eines Kick-off-Meetings weiter erläutert und detaillierter besprochen (siehe Anhang: Protokoll Kick-off-Meeting). Dabei wurde der Fokus insbesondere auf die nicht-technischen Aspekte des Projekts gelegt – wie Benutzerfreundlichkeit und die Förderung kollaborativer Arbeitsweisen – während technische Details und Funktionen bewusst reduziert wurden.\\

Die Vision, die aus diesem Meeting hervorging, ist die Entwicklung einer möglichst intuitiven Software, welche es auch fachfremden Personen erlaubt, schnell und unkompliziert einfache, schematische Skizzen zur räumlichen Planung zu erstellen.
\pagebreak

\subsection{Workshops}
Im Rahmen des Projekts haben wir einen Workshop des SCDH besucht. Dieser fand am 3. April 2025 am Standort Nidau statt und wurde vom Wohnheim Humanitas aus Horgen in Auftrag gegeben. Ziel des Workshops war es, die geplanten Arbeitsabläufe im zukünftigen Neubau des Wohnheims in Zusammenhang mit einem Assistenzkran zu testen.

Hierzu wurden die Baupläne auf die 1:1-Projektionsfläche übertragen. Das Team des Wohnheims nutzte diese Fläche, um verschiedene alltägliche Abläufe realitätsnah zu simulieren und zu evaluieren.\\

Der Workshop gliederte sich grob in drei Phasen:
\begin{itemize}
  \item Briefing
  \item Simulation auf der Projektionsfläche
  \item Debriefing
\end{itemize}

\includegraphics[]{workshop-storyboard.png}\\

Unser besonderes Augenmerk lag auf den Phasen \textit{Briefing} und \textit{Debriefing}, da diese im Kontext unseres Projekts besonders relevant sind. In diesen Abschnitten konnten wir zentrale Anforderungen aus der Aufgabenstellung validieren. Dabei wurde deutlich, dass unsere Lösung über das ursprüngliche Konzept hinaus weiteres Potenzial bietet – insbesondere im Hinblick auf die Unterstützung von Workshops (siehe Kapitel 3.1 und 3.2).
